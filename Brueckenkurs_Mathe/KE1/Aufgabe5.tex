Gegeben seien die Mengen: 

\begin{eqnarray}
	A = \left\{ \left(x,y \right) \in \mathbb{R} \times \mathbb{R}: x^2 +y^2 \le 16 \right\}\\
	B = \left\{ \left(x,y \right) \in \mathbb{R_{+}} \times \mathbb{R_{+}}: x > \sqrt{y} \right\}\\
	C = \left\{ \left(x,y \right) \in \mathbb{R} \times \mathbb{R}: y > -\dfrac{1}{2}x+3 \right\}\\
	D = \left\{ \left(x,y \right) \in \mathbb{R} \times \mathbb{R}: y \ge x^2 \right\}
\end{eqnarray}
Die Ermittlung der Hilfspunkte sind in Aufgabe 5b zu finden.