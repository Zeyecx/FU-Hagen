Bestimmen Sie die Menge: 
\begin{eqnarray}
	\left(A \cap B \cap C \cap D \right) \times \left(\mathbb{N}_0 \times \mathbb{N}_0 \right)
\end{eqnarray}

Lösung + Punkte aus 5a: \\


Definition $\mathbb{N}_0$:
\begin{eqnarray}
	N_0 = \left\{x \in \mathbb{Z} \vert x \ge 0 \right\}
\end{eqnarray}

Allgemein gilt daher:
\begin{eqnarray}
	\left(A \cap B \cap C \cap D \right)  = E \vert  \text{~~~aus a~}\\
	\left(A \cap B \cap C \cap D \right) \times \left(N_0 \times N_0 \right)\\
	E \times (\{x\in \mathbb{Z}\vert x \ge 0\}\times \{x\in \mathbb{Z}\vert x \ge 0\}) 
\end{eqnarray}

Betrachten wir nun $A$ im Zahlenbereich $\mathbb{N}_0$ fallen uns folgende Tupel auf: \\
\begin{eqnarray}
	\left\{ (-4,0);(0,4); (0,-4); (4,0)\right\}\in A_{\mathbb{N}_0} 
\end{eqnarray}

Betrachten wir nun $B$ im Zahlenbereich $\mathbb{N}_0$ fällt auf, dass der Bereich der Menge durch die positive Y-Achse und durch den Term $\sqrt(x)$ nach unten beschränkt ist. \\

Betrachten wir nun $C$ im Zahlenbereich $\mathbb{N}_0$ fällt auf, dass der Bereich der Menge durch die positive Y-Achse und durch den Term $-\dfrac{1}{2}x+3$ nach unten beschränkt ist. \\

Betrachten wir nun $D$ im Zahlenbereich $\mathbb{N}_0$ fällt auf, dass der Bereich der Menge durch die positive Y-Achse und durch den Term $x^2$ nach unten beschränkt ist.\\

Man kann dies Sich die gesuchte Flächte recht solide im Kopf vorstellen. $A$ stellt die Grundfläche ($\Omega$) dar. Diese wird dann durch die Mengen $A \cdots D$ begrenzt. So begrenzt $C$ von oben und $B$ und $D$ \An{rechts }und \An{links} \\

Die Abgeschlossene Menge hat folglich folgende Punkte in $N_0$:\\ 

\begin{eqnarray}
	E= \left\{ (1,3); (0,4) \right\} 
\end{eqnarray}

\clearpage

Für $(1,3)$ aus $\left(\mathbb{N}_0 \times \mathbb{N}_0\right)$:
\begin{itemize}
	\item A: $ x^2 +y^2 \rightarrow 1^2 + 3^2 \le 16$
	\item B: $ y > \sqrt{x} \rightarrow 3 > \sqrt{1}$
	\item C: $y > -\dfrac{1}{2}\cdot x+3 \rightarrow 3 > -\dfrac{1}{2} \cdot 1 +3 \rightarrow 3 > 2.5$
	\item D: $y \ge x^2 \rightarrow 3 \ge 1^2$
\end{itemize}
$\square$

\vspace{2cm}
Für $(0,4)$ aus $\left(\mathbb{N}_0 \times \mathbb{N}_0\right)$:
\begin{itemize}
	\item 
	A: $ x^2 +y^2 \rightarrow 0^2 + 4^2 \le 16$
	\item 
	B: $ y > \sqrt{x} \rightarrow 4 > \sqrt{0}$
	\item 
	C: $y > -\dfrac{1}{2}\cdot x+3 \rightarrow 4 > -\dfrac{1}{2} \cdot 0 +3 \rightarrow 4 > 3$
	\item
	D: $y \ge x^2 \rightarrow 4 \ge 0^2 $ 
\end{itemize}
$\square$